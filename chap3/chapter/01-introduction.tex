Values are personal beliefs about what is important in individuals' lives and therefore characterize preferences.\footnote{Values differ from personality traits. Personality traits describe how individuals behave across time and situations, while values refer to what they consider important. See \citet{Schwartz2012Overview} for a discussion on how values relate to attitudes, beliefs, traits, and norms.}
For instance, universalism is a value which all of us hold to a certain extent; this, in turn, influences our preferences for redistribution (\citealt{Enke2020Moral}).
One can think of as many values as there are preferences (e.g. for leisure or for fertility).
Studying the dynamics of values is therefore crucial to understand differences in preferences between economic agents which explain differences in behaviors (e.g. effort or fertility decisions), hence, gaps in economic outcomes (e.g. wage or employment).
Although the inter-generational transmission is key in explaining the formation of values, their subsequent dynamics are driven by life experiences.

As the saying goes: ``a wise man changes his mind, a fool never will''. What the saying does not explain is why the wise man began to reassess his mind. One potential answer would be that something happened to him, but if that something also happened to the fool, such an answer is not sufficient. 
Another avenue is to ask whether they pay the same costs to change their minds or not. In the latter case, the fool would not be so much of a fool. Although he is a fool, he may have friends with whom he shares values, hence changing his mind is costly as it creates a distance between him and them. One may argue that the wise man bears the same cost as he also has friends who share values with him. 
The key point of that riddle is that the two groups of friends are drastically different in the values they convey, thus, values are inter-dependent within groups and both---the fool and the wise man---aim to be consistent with respect to values held in their groups. Therefore, the life-changing event may have changed one of the wise man's values which made him less compatible with his friends' values, hence, he preferred to identify with a new group of friends and therefore changed all his values toward those of that new group.

This paper argues that because group identity is defined by a cluster of values, shocks to one value that induce a change in group membership will lead to changes in other values, hence creating \textit{spillover effects}. Individuals are social and form groups based on values they share with others. Whenever an event occurs in someone's life, this brings new information and can generate a shock on some of her values. This shock can drive the individual to identify with a new group---because the shocked values have become too distant from those of her previous group. By identifying with the new group, she changes all her values---including not initially affected values---toward those of the new group. By changing values that are not affected by the shock, life events generate spillover effects across values.

Based on social psychology, I develop a model where the dynamics of values is disciplined by two anchoring forces: \textit{time consistency} and \textit{group consistency}. 
The former indicates that one prefers her today's values being close to her yesterday's values, that is, that values be consistent over time. This induces rigidity shaping how values adjust over time after a life-changing event that brings new information. 
The latter relates to the proximity of values held within the group with which we identify, hence, one prefers values to be consistent with those of her group. 
Both consistencies are based on the concept of cognitive dissonance introduced by \citet{Festinger1957Theory} as individuals seek to avoid the psychological burden of having values that are dissonant with either their past self or their group.

Following a life-changing event, the agent faces a consistency trade-off between time consistency and group consistency. An event corresponds to an information shock on one of her values at the end of a period. 
In the next period, the individual has to reset her values subject to both time and group consistencies. 
One way to soften such a trade-off consists of diminishing the dissonance with her group by identifying with a new group (e.g. new friends or a new political party) which conveys values that are closer to her recently shocked value.
Thus, with endogenous group membership, the agent will consider identifying with another group, which may imply resetting all her values toward the ones of this new group. 
For this to occur, the information shock needs to be sufficiently large to make this costly convergence process more desirable than keeping the previous group identity.

When values are independent, the agent adjusts her shocked value \textit{independently} of other values by simply minimizing the distance between her past value (time consistency) and the value of the group to which she decides to belong (group consistency).
The inter-dependence between values distorts the consistency trade-off.
When values are correlated within groups, the agent adjusts all her values \textit{simultaneously} as the relative weight of both consistencies depends on the intensity of the inter-dependence between values.\footnote{The intensity of the inter-dependence between values is exogenous to the agent and reflects the mapping of values in the society; see \citet{Roccas2010Personal} for the importance of the cultural context.}
Thus, the trade-off is in favor of the group consistency as the dissonance with the current group occurs across several dimensions.
As a result, the information shock on one value that would lead the agent to identify with another group has to be larger than the one that is needed when values are independent. 
Yet, if such a shock occurs on a value, then the agent identifies with a new group and changes all her values toward the ones of the new group, hence, triggering the so-called \textit{spillover effect}.

I test the prediction of the theory about the existence of spillover effects by using data from two British cohort studies in which I measure individuals' values and observe political vote at several ages. Using a principal component analysis, I show that the variation in the answers to a large set of questions about values can be summarized by two main dimensions which will be the two values of my latter analysis. These two dimensions coincide with the (motivational types of) values introduced by \citet{Schwartz1992Universals, Schwartz2012Overview}.
% CONSERVATION VS OPENNESS TO CHANGE
The first dimension captures conservation versus openness to change---the preference for stability, security, tradition, and conformity versus the openness to new experiences related to self-direction and stimulation. For ease of exposition, in what follows, I refer to those values as \textit{conservatism} versus \textit{progressivism}.
% SELF TRANSCENDENCE VERSUS SELF ENHANCEMENT
The second dimension reflects self-transcendence versus self-enhancement---values associated to care for and concern about others such as universalism and benevolence versus the self-interest and ambition linked to achievement and power. In what follows, I refer to them as \textit{collectivism} versus \textit{individualism}.

%% Political vote
I use the political vote of individuals at the general election to proxy their group membership. The mapping of voters is consistent with the two-dimensional value space across cohorts and periods. For instance, Conservative voters tend to have conservative and individualist values, whereas Labour voters are instead progressive and collectivist.

%% OLS
The identification of changes in values and group membership is challenging. 
I start by estimating separately the effect of two exogenous and non-reversible life events---to have a girl as a first child (conditional on having a baby), and to have ever had cancer---on both individuals' values. Individuals who went through one of those two life events tend to have more conservative values but there are no significant differences in collectivism. 
Then, I estimate the probability to vote for each political party at the general election according to changes in values since the previous period. Changes in values are associated with changes in the likelihood to vote for the political parties, hence, with changes in the probability to identify with a new group.

%% IV
To examine the presence of spillover effects, I instrument conservatism by the information shock associated to the life event and then I look at the impact on collectivism. A one-standard-deviation increase in conservatism induces an increase in individualism of about one third of a standard deviation. Using the first-stage regression to estimate the probability of voting for each political party also indicates that increasing conservatism promotes the probability of voting for right-wing political parties over left-wing ones. Thus, providing empirical evidence of the group membership as the underlying mechanism in explaining the existence of spillover effects.

% SEM
The identification relies on the assumption that each life event brings no information shock on collective values. The identification assumption may be violated for many life events. For instance, to have ever been unemployed is likely to bring information shocks on both values, hence, the spillover effects cannot be identified in that setting. To deal with the two-side effect of unemployment on values that threatens identification, I use a simultaneous equations model in which I instrument endogenous values with their own respective lags.\footnote{I also address the question of the endogeneity of the life-event with respect to values in the case of unemployment. From the theoretical framework, I derive an expression of this bias that is a scale multiplier of the direct and indirect effects, hence, of the total effect. I show that \textit{i)} the bias can affect the magnitude of the total effect without changing the qualitative result, \textit{ii)} it is still possible to provide a lower-bound estimate of the effect, and \textit{iii)} the bias does not change the relative share of the total effect that is due to the direct and the spillover effects.} Thus, the identification relies on symmetrical exclusion restrictions which assume that one value is not directly affected by the lag of the other value. Based on the simultaneous equations model, I can estimate and decompose the change in values due to the information shock (direct effect) and the change owing to spillover effects across values (indirect effect). 

% RESULT 1
My empirical analysis yields three main results. First, life events change values throughout the lifecycle. Both exogenous life-changing events---to have a girl as a first child and to have ever had cancer---increase conservative values, while to have never been unemployed make individuals more progressive. Collectivist values are fostered by both the latter event and having ever had cancer.

% RESULT 2
Second, changes in values are associated with changes in political voting, hence, group membership. On the one hand, when individuals become more conservative they also become more likely to vote for right-wing political parties (e.g. Conservative Party or UKIP) with respect to left-wing ones (e.g. Labour Party or Green Party). On the other hand, when individuals become more collectivist they shift their vote toward non-traditional political parties (e.g. Green Party, or UKIP) instead of traditional ones (i.e. Conservative Party and Labor Party).

% RESULT 3
Third, life events affect both values at the same time since spillover effects across values do exist. After an increase in conservatism due to a life-changing event, collectivism declines by a third of the increase in conservatism. Once the framework is generalized to shocks that can simultaneously affect both values, the spillover effects become non-reciprocal: an increase in conservatism still generates a \textit{negative} spillover effect on collectivism; but an increase in collectivism generates a \textit{positive} spillover effect on conservatism. Thus, there is a spiral pattern in the dynamics between values that can be rationalized by the dynamic underpinnings of value changes from the social psychology literature (\citealt{Schwartz2012Overview}).

% CONTRIB 1
This paper is the first to show the existence of spillover effects across values by considering the multi-dimensionality of values that characterizes group identity as a cluster of values. Prior work analyses the dynamics of values but focuses on the evolution of a single value (\citealt{Piketty1995Social}, \citealt{Mayda2006Against}, \citealt{Fernandez2007Women}, \citealt{Alesina2018Intergenerational}, i.a.). I contribute to this literature by showing that neglecting the inter-dependence between values---i.e. assuming that values are independent---underestimates to which extent life experiences affect individuals because this omits the consequences of the group membership, hence, the spillover effects.

This paper adds to the literature on the formation and dynamics of values. Prior work highlights several mechanisms such as the inter-generational transmission (\citealt{Bisin2001Economics, Bisin2011Economics}, \citealt{Montgomery2010Intergenerational}, \citealt{Hiller2016Cultural}, \citealt{Alan2017Transmission}, i.a.) along with the role of cultural values (\citealt{Ichino2000Work}, \citealt{Fernandez2004Mothers}, \citealt{Guiso2006Culture}, \citealt{Fernandez2007Women}, \citealt{Giuliano2007Living}, \citealt{Chen2013Effect}, \citealt{Alesina2014Family}) and norms (\citealt{Fehr2002Psychological}, \citealt{Bardi2003Values}, \citealt{Tabellini2008Scope}) to explain how people form their values. Recent work focuses on the development of values during childhood (\citealt{Fehr2013Development}, \citealt{Doepke2017Parenting}, \citealt{Basic2020Development}).
I contribute to this literature by providing an additional mechanism based on cognitive dissonance and endogenous group membership (i.e. identity).

My work is also related to the literature on the consequences of cognitive dissonance in economics (\citealt{Akerlof1982Cognitive}, \citealt{Konow2000Fair}, \citealt{Benabou2006Belief}).
Prior work uses the concept of cognitive dissonance---introduced by \citet{Festinger1957Theory} and \citet{McGuire1960Cognitive}---to explain the belief-behavior relationship. I, instead, consider its effects on the between-values relationship; either to avoid dissonance with the previous self (\citealt{Eyster2002Rationalizing}, \citealt{Yariv2002See}) or to avoid dissonance with the values of the group.

My approach is also inspired by the literature on identity in economics (\citealt{Akerlof2005Identity, Akerlof2010Identity}, \citealt{Benabou2011Identity}, \citealt{Kranton2016Identity}). Prior work shows the effect of group membership on individual behavior (\citealt{Charness2007Individual}, \citealt{Sutter2009Individual}). I link changes in values, hence spillover effects, to change in endogenous group membership. Thus, individuals decide with which group they prefer to identify by comparing their values with the ones held in these groups.
In the empirical part, I build my identification strategy of changes in group membership using political identity (\citealt{Shayo2009Model}, \citealt{Bonomi2021Identity}).

My work also builds an additional bridge between the social psychology literature and that in economics. Psychological determinants of economic behaviors have been mostly introduced through personality traits (\citealt{Borghans2008Economics}, \citealt{Almlund2011Personality}, \citealt{Ferguson2011Personality}, \citealt{Becker2012Relationship}, \citealt{Flinn2018Personality}, \citealt{Todd2020Dynamic}). 
The \textit{big-five} personality traits are quite stable over the lifecycle and therefore can hardly explain changes in individuals' decision-making process (\citealt{Terracciano2006Personality, Terracciano2010Intra}, \citealt{Cobb-Clark2012Stability}). Thus, I introduce motivational types of values \textit{à la} \citet{Schwartz1992Universals, Schwartz2012Overview} as novel determinants of economic behaviors, which are more volatile than personality traits because of the impact of life experiences (\citealt{Lonnqvist2011Personal}, \citealt{Daniel2021Changes}).
Yet, personality traits and values are related as they look at the same object, individuals, from different perspectives which are therefore complementary (\citealt{Caprara2009Mediational}, \citealt{Fischer2015Motivational}, \citealt{Parks2015Personality}).

Lastly, my results on the consequences of life-changing events relate to three additional literatures. First, to the literature on the impact of children's gender on their parents' views.
\citet{Washington2008Female} finds that congressmen become more progressive in their voting after having a daughter. I, instead, find that having a girl as a first child makes parents more conservative. I show that both results can be reconciled as I find that tertiary-educated parents become indeed more progressive after having a girl. This suggests that \citet{Washington2008Female} captures the effect of having a daughter at the top of the distribution since congressmen tend to be highly educated; whereas I capture the average effect.
\citet{Grinza2017Entry} argue that, when entering into parenthood, women shift toward more conservative views.\footnote{Similarly, \citet{Bolzendahl2004Feminist} and \citet{Cunningham2005Reciprocal} find that entry into parenthood reduces the support for egalitarian roles for women and men in families.}
I provide additional evidence to this literature by showing that the effect is all the more important when they have a daughter and that changes in values are larger for mothers than for fathers.

Second, my work also relates to the literature on the impact of cancer on employment.
\citet{Peteet2000Cancer} discusses the relationship between cancer and the meaning of work, in a context where the loss of occupational identity becomes a source of anxiety and depression. \citet{Moran2011Long} show that cancer survivors have lower employment rates and work fewer hours than other similarly aged adults which can be due to consequences on life purpose and limitations in the ability to work (\citealt{Short2005Employment, Short2008Work, Short2008Long}, \citealt{Bradley2002Breast, Bradley2005Short}, i.a.). I add to this literature by providing an underlying mechanism through which cancer has consequences for employment, hence, through changes in values.

Third, my results relate to the literature on unemployment scarring as they open another potential explanation for this phenomenon. Unemployment is known to have consequences on well-being and health (\citealt{Clark1994Unhappiness}, \citealt{Knabe2010Dissatisfied}, \citealt{Nordt2015Modelling}). Scarring emphasizes the depreciation of human capital and firm-specific skills as the main driver of future employment (\citealt{Arulampalam2001Unemployment}, \citealt{Clark2001Scarring}, \citealt{Gregg2005Wage}). I show that having ever been unemployed decreases individualism, thus, if the likelihood to find a job is an increasing function of individualist values, then my framework would provide a novel mechanism in which past unemployment could affect future employment through changes in values.

The remainder of the paper proceeds as follows. Section \ref{chap3-theoretical} presents the theoretical framework and emphasizes the role of inter-dependence between values and consistency. Section \ref{chap3-data} describes the cohort data, derives values, shows the mapping of political parties on the two-dimensional value space, and presents the life events that are used as information shocks in the empirical part. Section \ref{chap3-empirics} shows the presence of spillover effects using instrumental variable regressions. Section \ref{chap3-simultaneous} presents the simultaneous equations model to identify spillover effects when the information shock affects both values simultaneously, and then discusses the dynamics between values in light of the social psychology literature. Section \ref{chap3-conclusion} concludes.
