%\subsubsection{Decomposing the effect of parental income} \label{chap2-trade-off}

This appendix examines how the effect of parental income on the occupations of mature workers operates through its impact on both first and second period occupations. Our results indicate that, conditional on first period occupation, the role of parental income in determining occupational outcomes has increased. At the same time, as is clear from the regressions, initial occupations are also important to determine outcomes at age 42. In particular, those who started their careers in middling have a probability to move to high-paying occupations that is about 7 pp. higher than those who started in a low-paying occupation (see Table \ref{chap2-tab:proba-group5-cdt-short}). Similarly, entering the labour market in a middling occupation implies a likelihood to be in such an occupation at age 42 at least 20 pp. higher than entering in a low-paying job. We would hence like to assess to what extent parental income compensates for past occupations. 

We compare the probability to be in a middling occupation at age 42 for two individuals who started in different initial occupations, middling and low-paying, and compute the additional parental income that the latter would need to have in order to compensate the advantage given by starting work in a middling occupation. To do so, we define the ratio between the two probabilities
\begin{equation}\label{chap2-eq:relative-proba}
    \frac{p_{M}^{M}}{p_{M}^{L}} = \frac{p_O^M}{p_O^L}\exp\Big(\eta_{MM} -\eta_{ML} + \beta_{3M} (Y^M-Y^L)\Big),
\end{equation}
where $p^j_k$ is the probability of being in occupation $k$ in second period conditional on having started in occupation $j$, and $Y^L$ and $Y^M$ are, respectively, the parental income of the individual starting in a low-paying occupation and of that starting in a middling occupation. 

For both cohorts, we derive the parental income $Y^c=Y^L-Y^M$ such that the two individuals are as likely to be in a middling occupation at age 42, i.e. $p_{M}^{M}=p_{M}^{L}$. Thus,
\begin{equation}\label{chap2-eq:counter-Y}
    Y^c \equiv \frac{\eta^c}{\beta^c} = \frac{\eta^c_{MM} -\eta^c_{ML} - \log(p_O^M/p_O^L)}{\beta^c_{3M}},
\end{equation}
where $p_O^M$ and $p_O^L$ are evaluated at the mean of the parental income distribution. We interpret $Y^c$ as the additional parental income that an individual in cohort $c$ starting in occupation $L$ needs in order to be as likely as one starting in occupation $M$ to be in a middling occupation when mature.
Thus, $\eta^c$ captures the degree of persistence in $M$, whereas $\beta^c$ reflects the effectiveness of parental income in moving into a middling occupation in second period. The greater the degree of persistence in $M$, the greater the parental income required to compensate the advantage given by the first-period occupation, i.e. $\partial Y^c/\partial\eta^c > 0$. The greater the effectiveness of parental income, the smaller the parental income required to compensate, i.e. $\partial Y^c/\partial\beta^c < 0~\forall\Delta\beta > 0$.

This difference in parental income reflects the value conferred by being in a certain first-period occupation---compared to parental income---for mobility across occupations. Taking the ratio between $Y^{70}$ and $Y^{58}$, we obtain the \emph{change across cohorts in the relative advantage} such that
\begin{equation}\label{chap2-eq:delta-Y}
    \Delta Y \equiv \frac{Y^{70}}{Y^{58}}= \frac{\Delta\eta}{\Delta\beta}
\end{equation}
where $\Delta\eta = \eta^{70}/\eta^{58}$ captures the effect of the change in the degree of persistence and $\Delta\beta = \beta^{70}/\beta^{58}$ reflects the effect of the change in the role of parental income. 
Both affect the worth of the first-period occupation in opposite ways.
%\footnote{ 
The greater the change in the degree of persistence, the greater the change in parental income needed to compensate, hence, the greater the relative worth of first-period occupation, i.e. $\partial\Delta Y/\partial\Delta\eta > 0$.
The greater the change in the effectiveness of parental income, the smaller the change in the parental income to compensate, hence, the smaller the relative worth of first-period occupation, i.e. $\partial\Delta Y/\partial\Delta\beta < 0~\forall\Delta\beta > 0$.
%}

Table \ref{chap2-tab:decomp-all} presents the decomposition of the relative advantage of first-period occupation---compared to parental income---for upward mobility.
\begin{table}[!tb]
    \centering
    \caption{Relative advantage of the first-period occupation with respect to parental income}
    \label{chap2-tab:decomp-all}
    \begin{threeparttable}
        \setlength{\tabcolsep}{9pt}
        
\begin{tabular}{lrrrrrrrrr}
\toprule
\multicolumn{1}{c}{} & \multicolumn{3}{c}{$p_{M}^{M}=p_{M}^{L}$} & \multicolumn{3}{c}{$p_{H}^{H}=p_{H}^{M}$} & \multicolumn{3}{c}{$p_{H}^{H}=p_{H}^{L}$} \\
\cmidrule(l{3pt}r{3pt}){2-4} \cmidrule(l{3pt}r{3pt}){5-7} \cmidrule(l{3pt}r{3pt}){8-10}
  & \multicolumn{1}{c}{$Y$} & \multicolumn{1}{c}{$\eta$} & \multicolumn{1}{c}{$\beta$} & \multicolumn{1}{c}{$Y$} & \multicolumn{1}{c}{$\eta$} & \multicolumn{1}{c}{$\beta$} & \multicolumn{1}{c}{$Y$} & \multicolumn{1}{c}{$\eta$} & \multicolumn{1}{c}{$\beta$}\\
\midrule
BCS70 & 4.63 & 0.73 & 0.16 & 2.13 & 0.84 & 0.39 & 2.25 & 0.89 & 0.39\\
NCDS58 & 13.11 & 0.60 & 0.05 & 5.47 & 0.79 & 0.14 & 6.24 & 0.90 & 0.14\\
\midrule
$\Delta$ & 0.35 & 1.22 & 3.45 & 0.39 & 1.07 & 2.74 & 0.36 & 0.99 & 2.74\\
\midrule
$\Delta$ ($\eta$ constant) & 0.29 & 1.00 & 3.45 & 0.37 & 1.00 & 2.74 & 0.37 & 1.00 & 2.74\\
$\Delta$ ($\beta$ constant) & 1.22 & 1.22 & 1.00 & 1.07 & 1.07 & 1.00 & 0.99 & 0.99 & 1.00\\
\bottomrule
\end{tabular}

        \begin{tablenotes}[flushleft]
            \footnotesize{\item\textit{Notes}: This table presents the relative advantage of the first-period occupation with respect to parental income for upward mobility. 
            $Y$ corresponds to the parental income that an individual in cohort $c$ needs in order to compensate for having started one occupational category below, $\eta$ captures the degree of persistence, whereas $\beta$ captures the effectiveness of parental income. 
            Coefficients for the NCDS58 and BCS70 cohorts are computed for males using Table \ref{chap2-tab:occ-multi23-base} in the appendix.
            $\Delta$ rows refer to the ratio between the BCS70 and NCDS58 under three specifications: the actual ratio, the ratio keeping $\eta$ constant, and the ratio keeping $\beta$ constant.}
        \end{tablenotes}
    \end{threeparttable}
\end{table}
We consider three cases: the difference in reaching a middling occupation for those starting in low-paying or in middling occupations (left panel), the difference in reaching a high-paying occupation for those starting in high-paying or in middling occupations (middle panel), and the difference in reaching a high-paying occupation for those starting in low-paying or in high-paying occupations (right panel). 

% BUT recall that parental income was ABSOLUTELY less important to determine first-period occ. in the NCDS58. Also eta has increased so the ABSOLUTE value of initial occupation has increased 

Consider first the relative effect of initial occupations versus parental income for the NCDS58. Because parental income is standardized, the figures reported for $Y^{70}$ represent the standard deviations needed to compensate the difference when starting in the various initial occupations (computed at the mean of parental income.). For the three cases we report, the additional income required is between 5.47 and 13.11 standard deviations. Such large magnitudes imply that it was hard for parental income to compensate the advantage conferred by a more favourable initial occupation, and, in the case of the probabilities of being in a middling occupation at age 42 ($p_M^M$  and $p_M^L$, left panel) only a massive difference in parental income could compensate the advantage that being in a middling occupation at 23 conferred. When we compare these figures with those for the BCS70, we can see that the additional income required to compensate the most favourable occupation is between 2.13 and 4.63 standard deviations, magnitudes that amount to about a third of those needed for the older cohort. 

The bottom two lines allow us to understand what is driving this change. We compute the ratio $Y^{70}/Y^{58}$ by keeping constant, i.e. at the value it had for the NCDS58, either $\eta$ or $\beta$. Recall from equation (5) that $\eta$ captures the degree of persistence in an occupation, whereas $\beta$ reflects the advantage to move upwards conferred by parental income. The three cases we examine display the same pattern. When we keep $\eta$ constant we obtain a change in the relative importance of parental income that is very close to the actual one, indicating that changes in persistence have played a minor role. In contrast, keeping $\beta$ constant results in values of $\Delta Y$ that are around or above 1. That is, what is driving the differences across cohorts in the advantage that parental income affords relative to initial occupations is the direct effect of the former rather than any changes in persistence associated with the latter.

These results indicate that there has been a major change in the relative roles that entry jobs and parental background play in determining the occupational outcomes of mature individuals. For the older cohort, the advantage conferred by entry occupations could only be offset by vast amounts of parental income; for the younger one, the latter has become much more able to offset the career advantages conferred by early career experiences.
