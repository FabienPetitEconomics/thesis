This thesis explores the role of changing inter-generational dynamics in Economics with a focus on distributional outcomes. The first chapter explores the decline of the labor share in the context of the appearance of boomer cohorts in France and the US. I build a theoretical framework in which a generational conflict arises because young and old individuals have different income sources and opposite objectives in terms of public policy. I argue that the shift away from labor toward capital by firms is a consequence of changes in labor market institutions which are endogenously determined by the age structure of the population. The second chapter describes how polarization in the labor market has been accompanied by a decline in inter-generational social mobility in the UK. My co-authors and I compare two British cohorts that entered the labor market at two points in time that differed considerably in terms of the structure of employment. We find that the role of parental income has increased for social mobility. We suggest that understanding inter-generational dynamics requires considering how individuals move from their entry jobs into other employment categories, i.e. understanding intra-generational mobility. The third chapter examines the consequences of life-changing events on individuals’ values over the lifecycle. I develop a theoretical framework to explain how individuals adjust their values when those latter are inter-dependent and shocked by life events. Bringing the model to British cohort data, I show that life experiences change individuals’ values directly but also indirectly as spillover effects appear when individuals seek consistency in their values.

\vspace{0.5cm}
\noindent\textbf{Keywords}: Inter-generational dynamics; Labor share; Social mobility; Individuals' values\\
% \textbf{JEL classification}: