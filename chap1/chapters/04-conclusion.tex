%% SUMMARY OF PAPER IDEAS AND MODEL
A vast literature emphasizes the role of biased technical change and institutions to explain the shift from labor toward capital and therefore the decline of the labor share observed in several countries over the past few decades. This paper focuses upstream of these determinants and highlights the role of demography as a force that shapes labor market institutions and hence the allocation of factor incomes. These institutions define the rules of the game for wage bargaining between firms and workers. When a particular generation, such as the boomers, can change institutions in its favor, then these rules also change, which affects the allocation of income between capital and labor. This mechanism accounts for the indirect \textit{policy-mechanism} effect of demographic changes on the labor share that results from the inter-generational conflict when choosing the public policy. Besides, the age structure of the population also has a direct \textit{factor-accumulation} effect that occurs through the labor supply and capital stock. Both effects combined help understand the role of the boomers' cohort in the decline of the labor share in France and the United States.

This paper shows to which extent we should take into account changes in institutions, that are endogenously determined by the age structure of the population, to understand macroeconomic dynamics in the long run. Decomposing the direct factor-accumulation effect and the indirect policy-mechanism effect, I find that the latter is as important as the former in explaining how demographic dynamics affect the labor share. Thus, omitting this indirect mechanism, and more broadly supposing that institutions do not change in the long run, leads to underestimating the role of demography on the factor income distribution. In this regard, my results provide a new conceptual framework to examine demographic dynamics and institutions in future work.

%% IMPLICATIONS IN TERMS OF POLICY DEBATE
These results have implications in terms of current policy debates. On the one hand, several high-income countries have experienced aging of their population which has led to a debate about optimal public policy. In this respect, my results shed light on the consequences of demographic changes on the allocation of income between capital and labor. On the other hand, developing countries are witnessing large demographic changes and may experience the arrival of a generation such as the boomers' cohort, which would change their institutions along with factor shares, and therefore, may have consequences on their development.